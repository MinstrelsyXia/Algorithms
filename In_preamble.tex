\usepackage{fancyhdr}
\pagestyle{fancy}
\fancyhead[L]{\slshape{Charlotte Xia}}

% theorems
\usepackage{thmtools}
\usepackage{thm-restate}
\usepackage[framemethod=TikZ]{mdframed}
\mdfsetup{skipabove=1em,skipbelow=0em, innertopmargin=12pt, innerbottommargin=8pt}

\theoremstyle{definition}

\declaretheoremstyle[headfont=\bfseries\sffamily, bodyfont=\normalfont,
	mdframed={
		nobreak,
		backgroundcolor=brown!14,
		topline=false,
		rightline=false,
		leftline=true,
		bottomline=false,
		linewidth=2pt,
		linecolor=brown!180,
	}
]{thmbrownbox}

\declaretheoremstyle[headfont=\bfseries\sffamily, bodyfont=\normalfont,
	mdframed={
		nobreak,
		backgroundcolor=Blue!4,
		topline=false,
		rightline=false,
		leftline=true,
		bottomline=false,
		linewidth=2pt,
		linecolor=NavyBlue!120,
	}
]{thmbluebox}

\declaretheoremstyle[headfont=\bfseries\sffamily, bodyfont=\normalfont,
	mdframed={
		nobreak,
		backgroundcolor=Green!5,
		topline=false,
		rightline=false,
		leftline=true,
		bottomline=false,
		linewidth=2pt,
		linecolor=OliveGreen!120,
	}
]{thmgreenbox}

\declaretheoremstyle[headfont=\bfseries\sffamily, bodyfont=\normalfont,
	mdframed={
		nobreak,
		topline=false,
		rightline=false,
		leftline=true,
		bottomline=false,
		linewidth=2pt,
		linecolor=OliveGreen!120,
	}
]{thmgreenline}

\declaretheoremstyle[headfont=\bfseries\sffamily, bodyfont=\normalfont,
	mdframed={
		nobreak,
		topline=false,
		rightline=false,
		leftline=true,
		bottomline=false,
		linewidth=2pt,
		linecolor=NavyBlue!70,
	}
]{thmblueline}

\declaretheorem[numberwithin=section, style=thmbrownbox, name={\color{Brown}Definition}]{defi}
\declaretheorem[numberwithin=section, style=thmgreenbox, name={\color{OliveGreen}Law}]{law}
\declaretheorem[numberwithin=section, style=thmbluebox, name={\color{Blue}Corollary}]{cor}
% 可以由Theorem直接推出来的东西
\declaretheorem[numberwithin=section, style=thmgreenline, name={\color{OliveGreen}Property}]{prt} 
\declaretheorem[numberwithin=section, style=thmbluebox, name={\color{Blue}Proposition}]{prp} 
% 一个比较特例的内容,它可能只对我们这篇文章里面的东西有用,也许是某个一般性性质的特殊应用,也许是专门要计算这么一部分内容,总之不一定适合给别人引用,没有稍微宽一点的普适性(但是Lemma可以有)。感觉是比较综合Lemma,Corollary和Remark的性质。
\declaretheorem[numberwithin=section, style=thmbluebox, name={\color{Blue}Theorem}]{thm} 
% 全文中最重要的(几个)内容,我写这篇文章就是要告诉读者这件事。也可以是某一章最重要的内容,说这一章我们就干这件事了,放在全文可能是main Theorem的一步。
\declaretheorem[numberwithin=section, style=thmbluebox, name={\color{Blue}Lemma}]{lemma}
% 和Theorem比没有那么重要,但是对于我们现在关注的Theorem有直接的作用,比如是拆解Theorem成几个不同的方向,后面要合起来,方便读者理解。
\declaretheorem[numberwithin=section, style=thmbrownbox,  name={\color{NavyBrown}Example}]{eg}
\declaretheorem[numberwithin=section, style=thmgreenline, name={\color{OliveGreen}Remark}]{remark}
\declaretheorem[numbered=no,style=thmblueline, name={\color{NavyBlue!70}Proof},qed=$\square$]{prf}